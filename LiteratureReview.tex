\section{Literature Review}

\subsection{Level Set Methods}
The level set method is a mature numerical method first proposed by Osher and Sethian in their seminal work\cite{osher_fronts_1988}. The method has  since been applied for a wide variety of applications: from image-processing and modeling flames to multiphase flows, and was first used for modeling flow in porous media by Prodanovi\'{c} and Bryant \cite{prodono}. The method was used for simulating drainage and imbibition in a porous medium of arbitrary geometry, when the contact angle is zero. By defining the location and propagation of the interface in an implicit manner, the level set method automatically handles operations such as interface splitting and merging. Hence it is ideal for tracking movement of an interface in a porous medium where phenomena like snap-off and trapping take place. Doing this using an explicitly defined interface, such as by front tracking, would be far more time consuming \cite{unverdi1992front}. A much more elaborate discussion of the level set method and the algorithms it uses can be found in the books by Osher and Sethian on the topic \cite{osher_level_2003, sethian_level_1999}. The level set method has already been in use for two-phase flow applications for incompressible fluid flow\cite{olsson_conservative_2005}. 

The level set method has also been modified for media of arbitrary wettability. In the formulation above, a zero contact angle is implicitly assumed. Zhao \emph{et al.} \cite{zhao_capturing_1998} proposed a variational approach for problems involving solid and fluid domains with different surface and bulk energies. The variational formulation for interface movement in porous materials has been implemented in a recent paper by Jettestuen \emph{et al.} \cite{jettestuen_level_2013}. The level set method can also be extended for modeling flow of more than two phases, for example by representing each interface by its own level set function \cite{esodoglu_variational_2008}. 

Level set methods suffer from problems of mass loss in under resolved regions. Enright \emph{et al.} proposed a new hybrid particle level set method which addresses this problem. This problem has also been tackled by other workers, such as by Sussman and Fatemi \cite{} who constrained the reinitialization step required in level set methods to incorporate area/volume conservation. Rider and Kothe \cite{} devised a set of test problems for determining the reliability of level set and related methods like front tracking. They found that Lagrangian tracking schemes maintain filamentary interface structures better than their Eulerian counterparts. Therefore, one of the objectives in this work is to compare different interface tracking methods and choose the one which is most accurate.

\subsection{Navier-Stokes Solvers: OpenFOAM VOF}

In this work, we focus on use of the volume of fluid (VOF) method for multiphase flow simulation using the OpenFOAM software tool. The VOF method has been around for some time, and is popular in the computational fluid dynamics (CFD) community to study interface motion in general fluid flow. Essentially, this is based on the idea of solving the Navier-Stokes equations for two fluids simultaneously. However, this would lead to solving for moving boundary conditions - solving which would be a very complex and time-consuming task. To get around this problem, the VOF method was introduced by Hirt and Nichols in 1981 \cite{hirt1981volume}. Essentially, instead of solving two sets of Navier-Stokes equations and keeping track of the fluid domain and shapes, we define an indicator function which contains the information of which fluid is contained in a given fluid cell. Huang \emph{et al.} (2005) used this method to study fluid flow in fractures, and were able to obtain qualitative agreements with laboratory experiments. In addition to VOF methods, multiphase flow using Navier-Stokes discretizations may also be done using the level set method \cite{} and phase field variables \cite{}. In this work, we will not focus on the phase field variable, and the level set method used is not coupled to the Navier-Stokes equations.

Deshpande \emph{et al.} present details of the numerical implementation of the VOF method in the OpenFOAM module interFoam \cite{}. Hoang \emph{et al.} (2013) performed OpenFOAM VOF simulations for confined bubbles and droplets in microfluidics. This is similar to the size range we are interested in. They make recommendations for optimal computational settings for various cases. They model surface tension dominated systems which is what we are interested in. Horgue \emph{et al.} (2013) used the VOF method to model two-phase flow in arrays of cylinders. They studied the spreading of a liquid jet in passages of size close to the capillary length. Ferrari and Lunati (2013) performed numerical drainage experiments to model the transition from stable flow to viscous fingering. They additionally evaluated different definitions of macroscopic and microscopic capillary pressures and conclude a definition based on the total surface energy provides an accurate estimate of the macroscopic capillary pressure.

OpenFOAM has not been very widely used for flow simulation in porous media at the pore scale. A major reason for that is the fact we are interested in modeling interface motion, which is typically hard to do in Navier-Stokes based solvers. The interfacial tension force may be applied using several approaches: continuous surface stress \cite{}, continuous surface force \cite{} or sharp surface force\cite{}. Specifically, for very low capillary numbers, large spurious velocities appear on the interface of the two fluids \cite{}. Lafaurie \emph{et al.} \cite{}proposed a relationship for estimating the magnitude of the spurious currents. Since we are interested in that domain of flow, this is a big problem for us.  Many works related to reducing these spurious currents also exist \cite{}. One recent work on this is Raeini \emph{et al.} (2012) \cite{}. They modeled the volume of fluid (VOF) method in order to simulate two-phase flow. The authors suggest a new filtering technique for removing these velocities from the interface and were able to get a much better resolution of the interface. In Raeni \emph{et al.} \cite{} they followed up their work with another paper applying their new technique to simulate two-phase flow in simple geometries like a star-shaped channel. They studied events like snap-off and layer flow and investigated the effect of geometry and flow rate on trapping and mobilization of disconnected blobs of fluid. In Raeini \emph{et al.} (2014) \cite{} they applied their method in a sandpack and a Berea sandstone. They proposed upscaling relationships for converting pore-scale pressures to Darcy-scale pressures, and derived relative permeability curves from their simulations. They also varied the capillary number in their simulations, in order to study capillary fingering versus viscous fingering. 

The problem of spurious velocities for low capillary numbers has been studied for a while, and many authors have proposed different approaches to deal with the problem. Renardy and Renardy (2002) \cite{} proposed a parabolic reconstruction method (PROST) for representation of the body force due to surface tension. Brackbill \emph{et al.} \cite{} also proposed an alternate technique to model surface tension as a body force.

The VOF method shares some similarities with the level set method. In both methods, a separate function indicates the fluid present at a particular point in the domain. For the volume of fluid methods, this would be a binary function - a zero value indicates one fluid, while 1 indicates another. For the level set function, the values change according to the distance from the interface. So, we have a gradation of values. Both methods update the function values using an advection equation. The form and implementation of this is discussed further in the Methods section. For now, suffice it to say that since the form of the equation is similar, similar methods to solve them exist. Direct comparisons of the two methods have also been done. Sussman and Puckett (2000) presented a hybrid level set-VOF method, and detailed problems and advantages for each method. Specifically, VOF methods are problematic for computation of local curvatures, since there is a sharp change in the indicator function value in the region of the interface. Standard VOF methods compute the curvature by smoothing the volume fractions in some way \cite{}. This is often problematic. If one does not smooth enough, even a simple circle would have oscillatory curvatures \cite{}. If one smooths too much, then the algorithm does not capture changes in curvature. A direct comparison of the level set and VOF methods will be made in this work, and preliminary results are shown in the Results section. Gerlach \emph{et al.} performed a direct comparison of three different techniques for surface tension modeling in VOF methods - Brackbill's continuum surface tension force formulation, the combined VOF and level set method proposed by Sussman and Puckett, and the PROST method proposed by Renardy and Renardy. They concluded that the PROST method was the most successful in eliminating these spurious currents.

VOF methods are typically computationally expensive. One strategy for speeding up convergence of this method is use of local time-stepping (LTS). The OpenFOAM module ltsinterFoam performs this function. Local time stepping speeds up convergence by using as large a time step as possible based on the local Courant number. This violates the local physics of the flow, but if one is only interested in equilibrium solutions, as is the case we are interested in, then this can be very useful \cite{}. The most restrictive time step in the interface region is spread through the domain by processing the time step field. Some smoothing is also required to prevent instability caused by large conservation errors due to large changes in time steps.

To our knowledge, the VOF method has not been applied for capillary dominated three phase flow. It has been applied for higher Reynold number flows, for example in work by Wardle and Weller \cite{} and Shonibare and Wardle \cite{}. This would be one of the objectives of this work. A direct simulation of three phase flow would not only be novel, but could also help verify analytical relationships for simple geometries proposed by other authors based on thermodynamics.

\subsection{Pore network models}
Representing pore space as a network of pores (openings) connected by throats was introduced by Fatt\cite{fatt1956network} in the 1950s. Later, in the 1980s, percolation theory was used to describe multiphase flow properties \cite{blunt_detailed_2002}. The history and development of the pore network modeling has been reviewed elsewhere \cite{blunt_flow_2001}. A few representative works are considered here. Jerauld \& Salter \cite{jerauld_effect_1990} describe a pore model to simulate two phase relative permeability and capillary pressure of strongly wetting systems (like Berea sandstone). The model was used to calculate scanning loops of hysteresis between primary drainage, imbibition and secondary drainage. Sahimi \& Helba \cite{helba_percolation_1992} was another paper which described the approach of using pore networks for generating flow properties, where they tried to develop a theory of two-phase flow using percolation theory. The earlier models were restricted to very simple geometries. Bryant \emph{et al.} \cite{bryant_prediction_1992}, used geologically realistic geometries, and for the first time were able to predict transport and flow properties. They based their models on a random close packing of equally-sized spheres. Diagenesis and compaction were modeled by adjusting the centers and radii of the spheres, thus making the model representative of sandstones. Single and multiphase flow was simulated through the pore space. They were able to predict the absolute and relative permeability, capillary pressure, and electrical and elastic properties of water-wet sand packs, sphere packs, and a Fontainebleau sandstone. {\O}ren \emph{et al.} \cite{oren_extending_1998} extended this work to simulate packing of spheres of different sizes. A reconstruction algorithm based on thin section analysis was used to generate a topologically disordered network. Two-phase flow was simulated and compared with experimental data. The model predicted the relative permeabilities for two water-wet sandstones and for a mixed-wet reservoir sample.

In the past couple of decades, more sophisticated ways of modeling the pore space have been developed: we can now simulate contact angles (and hence wettability), three-phases, non-Newtonian behavior, reactive flow solute transport and thermodynamically consistent oil layers \cite{van_dijke_existence_2006,oren2003reconstruction,blunt_detailed_2002,piri_three-phase_2004, piri_three-dimensional_2005}. One key aspect of pore network modeling is the extraction of networks representative of the pore space. The pores and throats are assigned effective properties, such as volume, inscribed radius and shape. Pore connectivity is also tracked, for example by grain-based approaches \cite{oren2003reconstruction}. Grain based approaches work well for clearly granular media and are well suited for pore space representations derived not from images, but from object based simulation of grain packing and diagenesis. They work less well for more complex systems, such as many carbonates, where grain identification is more difficult. There are other techniques for generation of the networks like the maximal ball approach \cite{silin2006pore} and the erosion-dilation technique \cite{lindquist2000pore, lindquist1999investigating} which work well for pore space geometries derived from images. 

In most network modeling approaches, the shape assigned to the pore is of great significance for proper wettability modeling. For demonstrating wetting layers, especially in three-phase flow, the wetting layer is modeled as being trapped in the corners, with the non-wetting phase in the centre of the pore. The simplest shape that captures this is a triangle. In three-phase flow, the intermediate wetting layer is sandwiched in between these two layers. Hence, the way the layers interact influences the relative permeability values. Many different shapes have been tried for simulating multiphase flow physics\cite{piri_three-phase_2004}.

Reviewing both direct modeling and network modeling approaches, Blunt \emph{et al.}\cite{blunt_pore-scale_2013}, conclude that network modeling still offers the most efficient and proven way of simulating multiphase flow in porous media, since workflows for network modeling are already mature. Direct simulation on representative samples of rocks (around 5 mm length scale) in some applications like mixed wettability and three-phase simulation is still beyond the reach of computers. Hence, direct simulation can at present only help to elucidate displacement mechanisms, and ultimately network modeling must be used to scale up the results using idealized geometries. Pore network modeling cannot lead to a good understanding of, for instance, overall connectivity of intermediate wetting layers in the network and their unusually high permeability due to somewhat ad-hoc rules of displacement \cite{dicarlo_three-phase_2000, dehghanpour_flow_2011}.


\subsection{Experimental and analytical data sets}

A large fraction of experimental observations at the pore scale have been micromodel experiments. Micromodels are idealized two-dimensional representations of porous media typically etched into a glass plate and monitored through a microscope, thus one can observe processes like fingering, snap-off and trapping associated with multiphase flow using them. For examples of micromodel experiments, see \cite{lenormand1989flow,lenormand1988numerical, lenormand1985invasion, mckellar1982method, soll1993micromodel}. Micromodels are a very popular method for observing multiphase as well as multicomponent flow behavior, including reactive flow \cite{corapcioglu1999glass, willingham2008evaluation}. Most of these models are fabricated, and hence are not exactly representative of real rocks - they need to have porosity of the order of 50\% in order to percolate, but are very useful for imaging flow. For instance, two and three phase displacement rules used in pore scale modeling have been devised based on micromodel observation \cite{oren_fluid_1995}.

High resolution X-ray micro-tomography experiments, first performed by Flannery \emph{et al.} \cite{flannery_three-dimensional_1987}, have enabled 3D multiphase observation. A review of X-ray computed tomography and its applications and limitations was done in Wildenschild \emph{et al.} \cite{wildenschild2002using}, and more recently in Wildenschild and Sheppard \cite{wildenschild2013x}. Recent flow imaging experiments were performed by Wildenschild and co-workers in a series of papers. Imaging of both 2 and 3 phase flow in packings of glass beads or general porous media is now common\cite{wildenschild_quantitative_2005, herring_effect_2013, piri_three-dimensional_2005, prodanovic_porous_2006}. In Culligan \emph{et al.} \cite{culligan_interfacial_2004}, the authors quantify the fluid-fluid interfacial areas using imaging of flow in bead packs. Mineralogy and dissolution/precipitation processes can be quantified as well \cite{cai2009tomographic}.

Focusing on the menisci formed during two-phase displacement, Mason and Morrow \cite{mason1986meniscus} determined the maximum meniscus curvatures (or critical curvatures) in rhomboidal pore throats formed by four ball bearings. In a later work \cite{mason_effect_1994}, they worked out critical curvatures for a range of contact angles and rhomboid pore angles, both analytically and experimentally. Their experiments were performed on PTFE ball packings.

There is of course a massive amount of literature present for experiments done at the lab scale\cite{homsy1987viscous, ingham2005transport, van1976mass}. Oak, 1990 \cite{oak1990three} measured three-phase relative permeability of Berea sandstone. However, we have not attempted to make any simulations beyond a single pore in this work.



