

\section{Research Objectives}


\subsection{ Develop fast and robust multi-phase capillary-dominated direct simulation technique}
	
Direct simulation of multi-phase flow in images can be done using numerical approximations to fluid flow equations. Popular methods for doing this are use of the Navier-Stokes (NS) based solvers (like OpenFOAM), and lattice-Boltzmann methods. The Lattice-Boltzmann methods in particular are popular for porous media flow since they can easily handle complex solid boundaries, which is an issue in Navier-Stokes based solvers, which rely on meshing of the fluid domain. The meshing process by itself is fairly complex and can be an issue for a general porous media image. When one encounters capillary-dominated multi-phase flow, tracking the interface becomes important. The most common for doing this in NS solvers are use of the volume of fluid (VOF) method. This technique is often too computationally expensive. Lattice-Boltzmann solvers are easy to implement as they can automatically handle complex solid boundaries. Hence one bypasses a major hurdle for simulation. These are more popular in the porous media community. However, they are also expensive computationally, but are set up to be intrinsically parallelizable. This makes them attractive methods for running on a large cluster. Various mutli-phase models exist for lattice Boltzmann solvers - some better than the others.

Both these broad class of methods suffer from inaccuracies in one form or the other. For example, the VOF method is known to have trouble capturing curvatures of fluid interfaces due to the way phase fractions are handled in the method. The level set method was initially introduced for precisely some of these problems. It works better than the VOF method for problems involving capture of fluid interfaces. It has been combined with the Navier-Stokes equations to yield good results in other works. However, it is also very expensive computationally and is difficult to parallelize.

In their 2006 paper, Prodanovic and Bryant proposed a simple level set model to capture capillary dominated fluid flow in porous media. They bypassed the Navier-Stokes approximations, and instead focused on quasi-static flow, and showed this approximation could be used to obtain good results with much lesser computational cost. Jettestuen and Helland in 2013 extended their work to impose contact angles at the boundaries. Verma (2014) carried out validation studies for a modified version of their method to validate the technique. If one could make the method more efficient, then large number of pores could be simulated cheaply. 

Strategies for speeding up the level set method already exist. One of the obvious steps is parallelization. Another one is implementation of adaptive meshing techniques. Both of these steps are under development.
	
\subsection{Validate techniques against experiments/analytical methods}
	
Validation of the contact angle level set method was already carried out for an experimental data set prepared by Mason and Morrow (1994). That data is for the critical drainage curvatures for a single pore in a rhomboidal packing of four spheres, for different contact angles. A related problem is mixed wettability. If some of the spheres have different contact angles, then one would get different critical curvatures. Mason and Morrow's work also presented an analytical technique for calculation of these curvatures for that geometry, which was essentially a modification of the Mayer-Stowe Princen (MS-P) method to calculate drainage curvature. This method would not work if the spheres have different wettabilities, for reasons elaborated upon later. 

Thus, we want to obtain an experimental data set similar to Mason and Morrow against which we could validate numerical techniques. These numerical techniques could then be used for larger porous media samples, and eventually matches against core-scale experimental data sets.
	
\subsection{Upscale results using pore network modeling}
Direct simulation techniques would always be computationally expensive, no matter how many speed up strategies we employ. Direct simulation on a sample of size, say of a core, would perhaps defeat the point of a digital approach. It is possible that performing the actual experiment in a laboratory is less time-consuming. However, pore network modeling can be used to simulate cores of large sizes, since it does not attempt to solve the Navier-Stokes equations in the pore space, but relies on simplified relationships for capillary displacement in each pore. However, the approximation usually relies on oversimplified geometries. Usually, this means some key aspects of the flow are not captured properly. Using direct simulation to obtain better rules for displacement of phases in general geometries should give better results for pore networks.

This procedure could be applied for both two and three phase modeling. An example of the application could be the behavior of sandwiched layers of oil between water and gas phases for water-alternating-gas injection processes. The state-of-the-art pore network modeling approaches fail to capture the behavior of the sandwiched layers accurately. It is quite possible the rules for layer stability are inadequate and direct modeling can help to get better rules.
